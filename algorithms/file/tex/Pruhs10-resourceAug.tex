%\documentclass[10pt,conference]{IEEEtran}

\documentclass[10 pt,final]{article}

\usepackage{amssymb} \usepackage{amsmath} \usepackage{amsthm} \usepackage{algorithm} \usepackage{algorithmic} \usepackage{url} \usepackage[margin=1in]{geometry}

\usepackage{subfigure}


\newtheorem{theorem}{Theorem} \newtheorem{lemma}{Lemma} \newtheorem{corollary}{Corollary} \newtheorem{definition}{Definition} \newtheorem{assumption}{Assumption} \newtheorem{example}{Example}
\newtheorem{observation}[theorem]{Observation}
%\newtheorem{theorem}{Theorem} \newtheorem{definition}{Definition} \newtheorem{remark}{Remark} \newtheorem{lemma}{Lemma} \newtheorem{corollary}{Corollary} \newtheorem{fact}{Fact} \newtheorem{invariant}{Invariant}
\usepackage{hyperref}
\usepackage{color}
\newcounter{todocounter}
\newcommand{\todo}[1]{\stepcounter{todocounter}\textcolor{red}{to-do\#\arabic{todocounter}: #1}}
\newcommand{\answer}[1]{{\color{magenta} #1}}
\newcommand{\question}[1]{{\color{blue} #1}}



\usepackage{graphicx}
\graphicspath{{./Figures/}}


\begin{document}
\title{Competitive Analysis using Resource Augmentation}
\date{}
\maketitle

%\author{Huangxin Wang\thanks{Department of Computer Science, George Mason University. Fairfax, VA 22030. Email: \textsf{hwang14@gmu.edu}}}
\href{https://github.com/hxwang/GreenDC-Summary/blob/master/papers/PruhsS10_schedule-energy.md}{Referred paper}


\section{Problem Model}
Jobs arrive over time at the data center, Job $i$ arrives at time $r_i$, with know work/size $w_i$ and known income function $I_i(t)$. The function $I_i(t)$, defined for all $t>0$, gives the income earned if job $i$ is completed at time $t$. We assume that the income function $I_i(t)$ is non-negative and non-increasing. We assume that the income goes to $0$ at the completion time approaches infinity, that is $\lim_{t \to \infty} I_i(t) = 0$.

The objective is to maximize the income.
\begin{align*}
\sum_i I_i(C_i) - E_i
\end{align*}
Notations used in this analysis is summarized in Table~\ref{tb_notation}.

\begin{table}[ht!]
\centering
\begin{tabular}{|l|l|}
\hline
notation & meaning \\ \hline
\hline
$r_i$ & job i's release time \\ \hline
$w_i$ & job i's size \\ \hline
$I_i(t)$ & income of completing job $i$ at time $t$ \\ \hline
$C_i$ & the completion time of job $i$ \\ \hline
$P(s_i)$ & the power consumption of running job $i$ at speed $s_i$ \\ \hline
$E_i$ & energy consumption of job $i$, $E_i = P(s_i)w_i/s_i$ \\ \hline
$n$ & total number of jobs \\ \hline
\end{tabular}
\label{tb_notation}
\end{table}



\section{Competitive Ratio Analysis}
\begin{lemma} The competitive ratio of any deterministic algorithm can not be bounded by any function of $n$ (total number of jobs), even if jobs have unit work.
\end{lemma}

\begin{proof}
We consider each processor has only two possible speeds, $1$ and $1/\epsilon$.

We consider two job instances and a power function for which $P(1) = 1$ and $P(1/\epsilon) = L/\epsilon$, where $\epsilon > 0$ is a small number and $L>1$. (For intuition, think of $L$ as large). 

\begin{itemize}
\item job 1 has $r_1 = 0, w_1=1$ and $I_i(t) = 1+\epsilon$ if $t \leq 1$ and $I_1(t)=0$ for $t > 1$.
\item job 2 has $r_2 = 1-\epsilon, w_2=1$ and $I_2(t) = L+1-\epsilon$ if $t\leq 1$ and $I_2(t) = 0$ for $t>1$.
\end{itemize}

In order to gain positive income by running job $1$, we need to run it at time $0$ and with speed $1$. If we run at speed $1/\epsilon$, then the income would be $1-L<0$. The profit of execute job $1$ at time $0$ is $1+\epsilon - 1 = \epsilon$. The profit of executing job $2$ at time $1-\epsilon$ is $L+1-\epsilon - L = 1-\epsilon$.

Denote the deterministic algorithm and adversary as ALG and ADV respectively.

\textbf{Case 1:} If ALG does not schedule job $1$, then ADV schedules job $1$ and does not release job $2$. Thus the competitive ratio is $\frac{ADV}{ALG} = \frac{\epsilon}{0} = \infty$. 

\textbf{Case 2:} If ALG schedule job $1$, then ADV release job $2$. Assume ALG does not switch to job $2$, then the profit it gains is $\epsilon$. Otherwise, the profit it gains is the income of job $2$ subtract the energy cost on job $1$, i.e., $1-\epsilon-(1-\epsilon)=0$. ADV will only execute job $2$ and gains income $1-\epsilon$. Thus the competitive ratio is $\frac{ADV}{ALG} = \frac{1-\epsilon}{\epsilon}$, by making $\epsilon$ small, we can make this ratio as large as we like.

\end{proof}


\section{Competitive Analysis with Resource Augmentation}

\begin{lemma} For any $\epsilon > 0$, there is an online algorithm $A$ that is $1+\epsilon$-speed $O(1/\epsilon^3)$-competitive for profit maximization.
\end{lemma}

\paragraph{$(1+\epsilon)$-speed: } If ADV can run at speed $s$ with power $P$, then ALG can run at speed $(1+\epsilon)*s$ with power $P$.

\subsection{Critical Speed} 

The maximum speed that job $i$ can be run without non-negative profit. Define the speed at $\hat{s_i}$, then we have

\begin{align}
I_i(t) - P(\hat{s_i}(t)) \frac{w_i}{\hat{s_i}(t)} = 0
\label{eq_critical-speed}
\end{align}

Divide Equation~\ref{eq_critical-speed} by $(1+\epsilon)$ and regrouping terms, we have
\begin{align}
I_i(t) - P(\hat{s_i}(t)) \frac{w_i}{(1+\epsilon)\hat{s_i}(t)} = \frac{\epsilon}{1+\epsilon} I_i(t)
\end{align}

\subsection{Online Algorithm}

\paragraph{High-level description}
\begin{itemize}
\item When a job arrives, we use the \emph{deadline setting} and \emph{job assignment policy} to set a deadline and assign the job to various time intervals on machines. Note, this assignment is not a schedule, as we many assign multiple jobs to the same machine at the same time.
\item We also set a speed via the speed scaling policy.
\item We then use \emph{job selection} policy to take some jobs from the intervals and machines on which they are assigned and actually run them on the machines.
\end{itemize}

Throughout this section, we use $\delta = \frac{\epsilon}{2}$.

\paragraph{Invariants} The online algorithm maintains a pool $Q$ of \emph{admitted} jobs. A job $i$ remains in Q until it is completed or its deadline passes. Each job $i$ in Q has several associated attributes.

\begin{itemize}
\item A deadline $d_i$ assigned to job $i$ when it was released.
\item The critical speed $\hat{s_i}^A = \hat{s_i}^A(d_i)$ derived from the deadline $d_i$ , and defined by Equation~\ref{eq_critical-speed}. The online algorithm will run job at speed $(1+2 \delta)\hat{s_i}^A$.
\item A collection of time intervals $J(i) = \{[t_1,t'_1], [t_2, t'_2], \cdots, [t_h, t'_h]\}$, where $r_i \leq t_1 \leq t'_1 \leq t_2 \leq \cdots \leq t_h \leq t'_h = d_i$. This collection $J(i)$ is fixed when job $i$ is released (but depends on previously scheduled jobs). The total length of the time intervals in $J(i)$ will be $(1+\delta) \frac{w_i}{(1+2\delta)\hat{s_i}^A}$.
\question{Question: why define the length in this way?}
\item A processor $m_{i,k}$ associated with job $i$ and each time interval $[j_k,j'_k] \in J(i)$ that was fixed at time $r_i$. Intuitively, at the time that job $i$ is released, the online algorithm is tentatively planning on running job $i$ on processor $m_{i,k}$ during the time period $[t_k, t'_k]$. We say that job $i$ is assigned to run on $m_k$ during times $[t_k, t'_k]$.
\end{itemize}

\paragraph{Deadline setting} Given a fixed deadline $d_i$, we can obtain the critical speed $\hat{s_i}^A = \hat{s_i}(d_i)$, the profit $p^A_i = I_i(d_i) - \frac{P(\hat{s}^A_i) w_i}{\hat{s}^A_i (1+2\delta)}$, and the online density $u^A_i = \frac{p_i}{t_i} = \frac{p_i \hat{s}^A_i}{w_i}$. 

Then let $c= 1 + \frac{2}{\delta}$, let $X(\frac{u^A_i}{c})$ be the set of jobs in Q with density at least $\frac{u^A_i}{c}$. Let $A$ be the maximum subintervals of $[r_i,t]$of times such that for each $[a,a'] \in A$, there is a processor $m_k$ for which no job in $X(\frac{u^A_i}{c}$ is assigned to run on $m_k$ during any time in $[a,a']$. Adjust $d_i$ to make $A$ be exactly the value $(1+\delta) \frac{w_i}{(1+2\delta)\hat{s_i}^A}$. If there is no $d_i$ that satisfy this condition, then the job will not be admitted.

\paragraph{Speed scaling policy} Every job is run at its critical speed for its set deadline.

\paragraph{Job selection policy} At any time $t$, on any processor $m_k$, run the job $i$, assigned to $m_k$ at time $t$, of maximum density.

\subsection{Construction of a structurally-nice near-optimal schedule OPT'}
We assume the optimal schedule is non-migratory. \question{Question: why make this assumption?}

To facilitate the comparison of online schedule to the optimal schedule, we modify the per-job power functions as follows, after modification, it will be more profitable.

\begin{align}
P'_i(s) = 
\begin{cases} 
P(s/(1+\epsilon)) & \mbox{if } s \in [s^{min}_i(C_i), \hat{s_i}(C^O_i)] \\ 
P(s) & \mbox{otherwise } . 
\end{cases}
\label{eq_modify-speed}
\end{align}

\begin{lemma} There exist a schedule that in $OPT'$ each job $i$ that runs does so at speed $(1+\epsilon)\hat{s_i}(C^O_i)$, and the total profit obtained using the modified power is at least $\frac{\epsilon}{1+\epsilon}$ times of the profit that OPT achieves using the power function $P$. 
\end{lemma}

\begin{proof}
$C^O_i$ is the completion time of job $i$ by OPT.
For OPT, it will run the job $i$ at speed $s \in [s^{min}_i, \hat{s}_i]$ where $s^{min}_i = s^{min}_i(C^O_i)$, and $\hat{s}_i = \hat{s}_i(C^O_i)$. Let OPT' run job at speed $(1+\epsilon)\hat{s}_i$, as we are speeding the job up, each job still completed by its completion time $C^O_i$. Now the net profit associated with job $i$ is
\begin{align*}
I_i(C^O_i) - P'_i(\hat{s_i}(1+\epsilon)) \frac{w_i}{\hat{s_i}(1+\epsilon)} = I_i(C^O_i) = P(\hat{s_i})\frac{w_i}{\hat{s_i}(1+\epsilon)} \\
= \frac{\epsilon}{1+\epsilon} I_i(C^O_i) \\
\ge \frac{\epsilon}{1+\epsilon} p^O_i
\end{align*}
\end{proof}

The first equality follows from the definition of $P'$, and the second equality follows from Equation~\ref{eq_modify-speed}, and the final equality follows because the income must be greater than the profit.

\subsection{Analysis of online algorithm}

%%%%%%%%%%%%%%%%%%%%%%%%%%%%%%%%%%%%%%%%%%%%%%%%%%%%%%%%%%%%%%%%%%%%%%%%%%%%%%%%





%\begin{figure}[!ht]
%\centering
%\subfigure[]{\includegraphics[width= 3 in]{TheoreticalProbability_1000Proxy.PNG}\label{lableA}}
%\subfigure[]{\includegraphics[width= 3 in]{BestAggressiveness_1000Proxy.pdf} \label{labelB}}
%\caption{
%(a)a caption,
%(b)b caption
%}
%\end{figure}


\bibliographystyle{plain}
\bibliography{FileNameOfBib}

%%%%%%%%%%%%%%%%%%%%%%%%%%%%%%%%%%%%%%%%%%%%%%%%%%%%%%%%%%%%%%%%%%%%%%%%%%%%%%%%

\end{document}

%%%%%%%%%%%%%%%%%%%%%%%%%%%%%%%%%%%%%%%%%%%%%%%%%%%%%%%%%%%%%%%%%%%%%%%%%%%%%%%%