%\documentclass[10pt,conference]{IEEEtran}

\documentclass[10 pt,final]{article}

\usepackage{amssymb} \usepackage{amsmath} \usepackage{amsthm} \usepackage{algorithm} \usepackage{algorithmic} \usepackage{url} \usepackage[margin=1in]{geometry}

\usepackage{subfigure}

\newtheorem{theorem}{Theorem} \newtheorem{lemma}{Lemma} \newtheorem{corollary}{Corollary} \newtheorem{definition}{Definition} \newtheorem{assumption}{Assumption} \newtheorem{example}{Example}
\newtheorem{observation}[theorem]{Observation}
%\newtheorem{theorem}{Theorem} \newtheorem{definition}{Definition} \newtheorem{remark}{Remark} \newtheorem{lemma}{Lemma} \newtheorem{corollary}{Corollary} \newtheorem{fact}{Fact} \newtheorem{invariant}{Invariant}

\usepackage{color}
\newcounter{todocounter}
\newcommand{\todo}[1]{\stepcounter{todocounter}\textcolor{red}{to-do\#\arabic{todocounter}: #1}}
\newcommand{\answer}[1]{{\color{magenta} #1}}
\newcommand{\question}[1]{{\color{blue} #1}}



\usepackage{graphicx}
\graphicspath{{./Figures/}}


\begin{document}
\title{Two Dimensional Cutting}
\date{}
\maketitle

%\author{Huangxin Wang\thanks{Department of Computer Science, George Mason University. Fairfax, VA 22030. Email: \textsf{hwang14@gmu.edu}}}

\section{Problem Model}
A large rectangle $A_0 = (L_0, W_0)$ of length $L_0$ and width $W_0$ is to be cut into m smaller rectangular pieces; piece i has size $(L_i, W_i)$ and value $v_i$. Let $P_i$ and $Q_i$ be the minimum and maximum number of pieces of type i that can be cut from $A_0 (0 \leq P_i \leq Q_i$ for $i=1,\cdots, m)$.

Notations used in this analysis is summarized in Table~\ref{tb_notation}.

\begin{table}[ht!]
\centering
\begin{tabular}{|l|l|}
\hline
notation & meaning \\ \hline
\hline
$A_0$ & a large rectangle \\ \hline
$L_0$ & length of the large rectangle \\ \hline
$W_0$ & width of the large rectangle \\ \hline
$L_i$ & length of type $i$ rectangle \\ \hline 
$W_i$ & width of type $i$ rectangle \\ \hline
$v_i$ & value of type $i$ rectangle \\ \hline 
$P_i$ & minimum number of pieces of type $i$ rectangle \\ \hline 
$Q_i$ & maximum number of pieces of type $i$ rectangle \\ \hline 
\end{tabular}
\label{tb_notation}
\end{table}


\section{Solution}
We define $q_{ipqr}$ and $x_{ipq}$ in the following

\begin{displaymath}
a_{ipqrs} = 
\begin{cases} 
1 &\mbox{if a piece of type i, when cut its bottom left-hand corner at (p,q), cuts out the point (r,s)} \\ 
0 & \mbox{otherwise} \\
\end{cases} 
\end{displaymath}

To prevent double counting when two pieces are cut adjacent to one another, we define
\begin{displaymath}
a_{ipqrs} = 
\begin{cases} 
1 &\mbox{if $0 \leq p \leq r \leq p+L_i-1 \leq L_0 - 1$ and $0 \leq q \leq s \leq q+W_i-1 \leq W_0 -1$} \\ 
0 & \mbox{otherwise} \\
\end{cases} 
\end{displaymath}

We define $x_{ipq}$

$a_{ipqrs}$ = \\
\begin{displaymath}
\begin{cases} 
1 &\mbox{if a piece of type i is cut with its bottom left-hand corner at (p,q) where $0 \leq p \leq L_0 - L_i$ and $0 \leq q \leq W_o - w_i$} \\ 
0 & \mbox{otherwise} \\
\end{cases} 
\end{displaymath}



Then the program is 
\begin{align}
\mbox{maximize } \sum^m_{i=1} \sum_{p \in L} \sum_{q \in W} v_i x_{ipq} \\
\mbox{subject to } \sum^m_{i=1} \sum_{p \in L} \sum_{q \in W} a_{ipqrs} x_{ipq} \leq 1, \forall r \in L, s \in W \\
P_i \leq \sum_{p \in L} \sum_{q \in W} x_{ipq} \leq Q_i, i = 1, ..., m \\ 
x_{ipq} \in (0,1), i=1,..., m ,\forall p \in L, q \in W 
\end{align}

The first constraint ensures that any point is cut out by at most one pieces;

The second constraint ensure that the number of cut pieces of any type lies within the required range;

The third constraint is the integrality constraint.
%%%%%%%%%%%%%%%%%%%%%%%%%%%%%%%%%%%%%%%%%%%%%%%%%%%%%%%%%%%%%%%%%%%%%%%%%%%%%%%%



%%%%%%%%%%%%%%%%%%%%%%%%%%%%%%%%%%%%%%%%%%%%%%%%%%%%%%%%%%%%%%%%%%%%%%%%%%%%%%%%





%\begin{figure}[!ht]
%\centering
%\subfigure[]{\includegraphics[width= 3 in]{TheoreticalProbability_1000Proxy.PNG}\label{lableA}}
%\subfigure[]{\includegraphics[width= 3 in]{BestAggressiveness_1000Proxy.pdf} \label{labelB}}
%\caption{
%(a)a caption,
%(b)b caption
%}
%\end{figure}


\bibliographystyle{plain}
\bibliography{FileNameOfBib}

%%%%%%%%%%%%%%%%%%%%%%%%%%%%%%%%%%%%%%%%%%%%%%%%%%%%%%%%%%%%%%%%%%%%%%%%%%%%%%%%

\end{document}

%%%%%%%%%%%%%%%%%%%%%%%%%%%%%%%%%%%%%%%%%%%%%%%%%%%%%%%%%%%%%%%%%%%%%%%%%%%%%%%%