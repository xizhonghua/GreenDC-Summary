%\documentclass[10pt,conference]{IEEEtran}

\documentclass[10 pt,final]{article}

\usepackage{amssymb} \usepackage{amsmath} \usepackage{amsthm} \usepackage{algorithm} \usepackage{algorithmic} \usepackage{url} \usepackage[margin=1in]{geometry}

\usepackage{subfigure}

\newtheorem{theorem}{Theorem} \newtheorem{lemma}{Lemma} \newtheorem{corollary}{Corollary} \newtheorem{definition}{Definition} \newtheorem{assumption}{Assumption} \newtheorem{example}{Example}
\newtheorem{observation}[theorem]{Observation}
%\newtheorem{theorem}{Theorem} \newtheorem{definition}{Definition} \newtheorem{remark}{Remark} \newtheorem{lemma}{Lemma} \newtheorem{corollary}{Corollary} \newtheorem{fact}{Fact} \newtheorem{invariant}{Invariant}

\usepackage{color}
\newcounter{todocounter}
\newcommand{\todo}[1]{\stepcounter{todocounter}\textcolor{red}{to-do\#\arabic{todocounter}: #1}}
\newcommand{\newadded}[1]{{\color{magenta} #1}}
\newcommand{\question}[1]{{\color{blue} #1}}



\usepackage{graphicx}
\graphicspath{{./Figures/}}


\begin{document}
\title{Competitive Ratio Analysis using Amortized Technique}
\date{}
\maketitle

%\author{Huangxin Wang\thanks{Department of Computer Science, George Mason University. Fairfax, VA 22030. Email: \textsf{hwang14@gmu.edu}}}

\section{Problem Model}
Assume there is a router (machine) with buffer to store packets. Assume packets arrive at the router over time to be send. Each packet $i$ has a release time $r_i$, a deadline $d_i$ and a value $v_i$. Sending a packet before its deadline will earn a revenue $v_i$. At each time slot $t$, at most one packet can be sent. The objective is to decide which packets to send in order to maximize the total revenue.

Notations used in this analysis is summarized in Table~\ref{tb_notation}.

\begin{table}[ht!]
\centering
\begin{tabular}{|l|l|}
\hline
notation & meaning \\ \hline
\hline
$r_i$ & packet $i$'s release time \\ \hline
$d_i$ & packet $i$'s deadline \\ \hline
$v_i$ & packet $i$'s value \\ \hline
$\mathcal{I}$ & the set of packets \\ \hline
$e$ & current earliest deadline packet \\ \hline
$h$ & current largest value packet \\ \hline
\end{tabular}
\label{tb_notation}
\end{table}



\section{Competitive Ratio Analysis}
\begin{lemma} The Greedy algorithm which always sends the packet with the largest value at each time slot has competitive ratio $2$.
\end{lemma}

\begin{proof} Denote the adversary and the greedy algorithm as ADV and ALG respectively. Our objective is to prove $\frac{ADV(\mathcal{I})}{ALG(\mathcal{I})} \leq 2$. We use induction to prove it. In specific, we show that two invariants will hold at each time slot t.
\begin{enumerate}
\item At each time t, the set of packets remaining in ADV's buffer $\mathcal{I}_{ADV}$ is a subset of the set of packets $\mathcal{I}_{ALG}$ in ALG's buffer, i.e., $\mathcal{I}_{ADV} \subset \mathcal{I}_{ALG}$.
\item At each time t, the revenue achieved by ADV and ALG satisfies $\frac{ADV}{ALG} \leq 2$.
\end{enumerate}



Initially, at time $0$, ADV and ALG have the same set of packets, and their revenues are both $0$, thus both invariants hold. 

\question{What's the initial set of packets? I assume initially both sets are empty.}

Assume at time $t$, these two invariants hold, in the following, we will show that after the execution of algorithm ADV and ALG, these two invariants still hold. 

Denote the earliest deadline packet and largest value packet in set $\mathcal{I}_{ALG}$ as $e$ and $h$ respectively. ALG will send the largest value packet $h$ at time slot $t$.

\textbf{Case 1:} Assume ADV also sends packet $h$ at time slot $t$, then $\mathcal{I}_{ADV} \setminus h \subset \mathcal{I}_{ALG} \setminus h$, invariant 2 holds. And $\frac{ADV}{ALG} = \frac{v_h}{v_h} = 1 \leq 2$, invariant 1 holds. Both invariants hold.

\textbf{Case 2:}Assume ADV sends packet $p (p!=h)$ at time slot $t$. Then we argue that ADV must will send $h$ at a later time. Otherwise, it can send $h$ rather than $p$ at current time $t$ and gains more revenue. Then using amortized analysis, we charge ADV the value of packet $p$ and $h$ at current time slot. Thus $\mathcal{I}_{ADV} \setminus \{h,p\} \subset \mathcal{I}_{ALG} \setminus h$, and $\frac{ADV}{ALG} = \frac{v_h+ v_p}{v_h} \leq 2$. Thus the two invariant still hold.
\end{proof}

\question{Using the amortized mechanism, we may charge ADV at time t for a later scheduled packet h. My question is under this approach, will there be a time that ALG schedules a packet, but ADV not. This could happen since we use amortized charging to charge a packet $h$ at an earlier time for ADV. }


%%%%%%%%%%%%%%%%%%%%%%%%%%%%%%%%%%%%%%%%%%%%%%%%%%%%%%%%%%%%%%%%%%%
\section{Competitive Ratio Analysis with Resource Augmentation}
\begin{lemma} If packets have aggreable deadline, and assume greedy algorithm has $2$ machines to send packets. Then the Greedy algorithm which always sends the packet with the largest value first is 1-competitive 2-speed.
\end{lemma}

\begin{proof} Again, we use the induction approach. The 2 invariants are:
\begin{enumerate}
\item At each time t, the set of packets remaining for ADV to schedule $\mathcal{I}_{ADV}$ is a subset of the set of packets $\mathcal{I}_{ALG}$ remaining for ALG, i.e., $\mathcal{I}_{ADV} \subset \mathcal{I}_{ALG}$.
\item At each time t, the revenue achieved by ADV and ALG satisfies $\frac{ADV}{ALG} \leq 1$.
\end{enumerate}

Initially, at time $0$, ADV and ALG have the same set of packets, and their revenues are both $0$, thus both invariants hold. 

Assume at time $t$, these two invariants hold, in the following, we will show that after the execution of algorithm ADV and ALG, these two invariants still hold. 

Let $e$ and $h$ denote the packets with the earliest deadline and the largest value respectively in the packet set $\mathcal{I}_{ALG}$. Then at time $t$, greedy will send both packet $e$ and $h$ using the two machines it has. 

\textbf{Case 1: } Assume ADV sends packet $h$. Then packet $e$ will not be in the buffer of ADV. Otherwise, packet $e$ will be sent first. Thus after time $t$, $\mathcal{I}_{ADV} \setminus \{h\}  \subset \mathcal{I}_{ALG} \setminus \{e,h\}$, invariant 1 holds. And $\frac{ADV}{ALG} = \frac{v_h}{v_h+v_e} < 1$, invariant 2 holds.

\textbf{Case 2:} Assume ADV sends packet $e$. Then it must will send packet $h$ at a later time, otherwise it can instead send packet $h$ rather than $e$ at this time step. Thus we use amortize mechanism to charge ADV at current time step for the later sent packet $h$, i.e., ADV will be charged for both packet $e$ and $h$ at time slot $t$. With similar analysis, these two invaraints still hold.

\textbf{Case 3:} Assume ADV sends packet $p (p!=e, p!=h)$. Then ADV must will send $h$ at a later time. If $e$ is scheduled at a later time, then we switch $e$ with $p$ and convert the problem to case $1$. Otherwise, assume the packet sending sequence of ADV is $p,s_1,s_2,\cdots,h$, then we change the sequence to $h,p,s_1,\cdots,s_n$. However, we need to guarantee that $s_1, \cdots, s_n$ will not miss their deadline. 

\question{What if the deadline of packet e, p, $s_1$ is 1, 2, 2 respectively. After the transfer, $s_1$ will miss its deadline?} 
\end{proof}



%%%%%%%%%%%%%%%%%%%%%%%%%%%%%%%%%%%%%%%%%%%%%%%%%%%%%%%%%%%%%%%%%%%%%%%%%%%%%%%%



%%%%%%%%%%%%%%%%%%%%%%%%%%%%%%%%%%%%%%%%%%%%%%%%%%%%%%%%%%%%%%%%%%%%%%%%%%%%%%%%





%\begin{figure}[!ht]
%\centering
%\subfigure[]{\includegraphics[width= 3 in]{TheoreticalProbability_1000Proxy.PNG}\label{lableA}}
%\subfigure[]{\includegraphics[width= 3 in]{BestAggressiveness_1000Proxy.pdf} \label{labelB}}
%\caption{
%(a)a caption,
%(b)b caption
%}
%\end{figure}


\bibliographystyle{plain}
\bibliography{FileNameOfBib}

%%%%%%%%%%%%%%%%%%%%%%%%%%%%%%%%%%%%%%%%%%%%%%%%%%%%%%%%%%%%%%%%%%%%%%%%%%%%%%%%

\end{document}

%%%%%%%%%%%%%%%%%%%%%%%%%%%%%%%%%%%%%%%%%%%%%%%%%%%%%%%%%%%%%%%%%%%%%%%%%%%%%%%%